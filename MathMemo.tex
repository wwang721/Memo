\documentclass{article}
\usepackage{ctex}
\usepackage{color}
\usepackage{amssymb}
\usepackage{amsmath}
\usepackage{geometry}
\geometry{a4paper,centering,scale=0.8}
\begin{document}
	\noindent \heiti\zihao{2}{数学备忘录}\zihao{-4}\songti\\ \hrule\bf{}
	\section{简单积分}
	$\int \sec x\mathrm{d}x=\ln |\sec x+\tan x|+C$
	
	$\int \csc x\mathrm{d}x=\ln |\csc x-\cot x|+C$
	\section{微分方程}
	\noindent\textcolor{red}{\S} 一阶线性微分方程:
	\[
		\frac{\mathrm{d}y}{\mathrm{d}x}+P(x)y=Q(x)
	\]
	若$Q(x)\equiv 0$,则为\textcolor{red}{\bf{齐次方程}},否则为\textcolor{red}{\bf{非齐次方程}}.
	\\方程通解为:
	\[
		y=\mathrm{e}^{-\int P(x)\mathrm{d}x} (\int Q(x)\mathrm{e}^{\int P(x)\mathrm{d}x}\mathrm{d}x+C)	
	\]
	\\[0.5cm]
	\noindent\textcolor{red}{\S} 二阶齐次线性微分方程:
	\[
		y''+py'+qy=0	
	\]
	可得到其特征方程:$r^2+pr+q=0\Rightarrow r_1$、$r_2$.
	
	\textcircled{1} $r_1\neq r_2\in\mathbb{R}$,则$y=C_1\mathrm{e}^{r_1x}+C_2\mathrm{e}^{r_2x}$
	
	\textcircled{2} $r_1= r_2\in\mathbb{R}$,则$y=(C_1+C_2x)\mathrm{e}^{rx}$
	
	\textcircled{3} $r_1=\alpha+\beta\mathrm{i},r_2=\alpha-\beta\mathrm{i}$,则$y=\mathrm{e}^{\alpha x}(C_1\cos\beta x+C_2\sin\beta x)$
	\\[1cm]
	\noindent\textcolor{red}{\S} 二阶非齐次线性微分方程:
	\[
		y''+py'+qy=f(x)
	\]
	
	\noindent<1>若$f(x)=\mathrm{e}^{\lambda x}P_m(x)\Rightarrow y^*=R(x)\mathrm{e}^{\lambda x}$
	\\
	
	\textcircled{1}若$\lambda$不是特征方程的根:\\
	\phantom{aaaaaaaaaaaaaaaaaaaaaa}$R(x)=R_m(x)$
	
	\textcircled{2}若$\lambda$是特征方程的单根:\\
	\phantom{aaaaaaaaaaaaaaaaaaaaaa}$R(x)=xR_m(x)$
	
	\textcircled{3}若$\lambda$是特征方程的重根:\\
	\phantom{aaaaaaaaaaaaaaaaaaaaaa}$R(x)=x^2R_m(x)$
	\\[1cm]
	\noindent<2>若$f(x)=\mathrm{e}^{\lambda x}[P_l(x)\cos\omega x+Q_n(x)\sin\omega x]$\\
	\phantom{aaaaaaaaaaaaaa}$\Downarrow$\\
	\phantom{aaaaaaaaa}$y^*=x^k\mathrm{e}^{\lambda x}[R_m^{(1)}\cos\omega x+R_m^{(2)}\sin\omega x],m=\max\{l,n\}$
	
	\textcircled{1}若$\lambda+\omega\mathrm{i}$不是特征方程的根$\Rightarrow k=0$.
	
	\textcircled{1}若$\lambda+\omega\mathrm{i}$是特征方程的单根$\Rightarrow k=1$.
	\\
	
	\noindent\textcolor{red}{\S} 欧拉方程:
	\[
		x^ny^{(n)}+p_1x^{n-1}y^{(n-1)}+\dots+p_{n-1}xy'+p_ny=f(x)	
	\]
	令$t=\ln x$即$x=\mathrm{e}^t$,可得$xy'=x\dfrac{\mathrm{d}y}{\mathrm{d}x}=\dfrac{\mathrm{d}y}{\mathrm{d}t}=Dy$\\
	则\[x^ky^{(k)}=D(D-1)\dots(D-k+1)y=\prod_{i=1}^{k}(D-i+1)y\]
	变形后即可解.(\textcolor{red}{注}:记得将$x$带回)
\end{document}
